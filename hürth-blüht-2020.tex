\documentclass[10pt]{article}
\usepackage{ngerman}
\usepackage[utf8]{inputenc}
\usepackage{graphicx}
\usepackage{subcaption}
\usepackage{hyperref}
\title{\textbf{Hürth-blüht Jahresbericht 2020}}
\author{Ivo Bathke\\
		BUND Hürth\\}
\date{}
\begin{document}

\maketitle

\begin{figure}[h!]
  \includegraphics[width=\linewidth]{img/titel.jpg}
\end{figure}

Mitwirkende:
\begin{itemize} 
\item BUND Hürth: Martyna Bartolome, Ivo Bathke, Andreas Palm, Sebastian Schöne  
\item Stadt Hürth
\item Stadtwerke Hürth 
\item Fotos: Ivo Bathke
\end{itemize}

\newpage

\section{Einführung}
Im nun zweiten Jahr von \textbf{Hürth-blüht} wollen wir die Entwicklung der Flächen betrachten und auch 4 neue Flächen vorstellen, die im Jahr 2020 angelegt wurden.


Im Frühjahr wurde bei einem Treffen von BUND, Stadt Hürth und Stadtwerken das Projekt besprochen und das Vorgehen über das Jahr geplant. Es wurde beschlossen, weitere Flächen auszusuchen und die Durchführung einer Streifenmahd. Dies bedeutet, dass bei der Mahd regelmäßig ein Streifen stehen gelassen wird, um Insekten weiterhin Nahrung und Deckung zu bieten.

Das Projekt \textbf{Hürth-blüht} wurde zum Bundeswettbewerb "`Naturstadt"' eingereicht, war jedoch leider nicht erfolgreich. Von insgesamt 322 eingereichten Projektideen wurden nur 40 Beiträge ausgewählt.

Die zunehmend trockenen Sommer sind auch für die Blühflächen eine Herausforderung. Insgesamt kommen die Flächen aber damit zurecht, deutlich besser als kurzgemähte Wiesen.
Es gibt jedoch auch Probleme an besonders trockenen Standorten, wie etwa am Hürther Bogen.

Die Resonanz aus der Bevölkerung war durchaus positiv. So wurde häufig beobachtet, wie Fotos von den Blühflächen gemacht wurden, auch wurden Blumensträuße gepflückt und manche Passanten wähnten sich bereits im Urlaub.

Die Blühflächen haben neben der Optik aber auch noch einen weiteren Grund: Dem Insektensterben und dem Verlust an Artenvielfalt entgegen zu treten. Die botanische Vielfalt konnte allein schon durch die Blühwiesen erhöht werden. Die Flächen wurden auch durch etliche Insekten angenommen, deutlich mehr als in den umliegenden gemähten Flächen. Jedoch meistens eher durch häufige Arten, vermisst werden weiterhin Arten wie: Blutströpfchen, Dickkopffalter und Großes Heupferd.


\newpage

\section{Burgpark}

\begin{figure}[h!]
  \includegraphics[width=\linewidth]{img/infotafeln.jpg}
  \caption{Burgpark}
\end{figure}

\begin{flushleft}
Zustand der Fläche ist sehr gut.

Der Effekt der Streifenmahd konnte hier gut beobachet werden.
So waren im ungemähten Abschnitt weiterhin viele Insekten zu sehen, während im gemähten Bereich die nächsten Pflanzenarten hochwuchsen.

Der neue Fahrradweg wird vermutl. eine Teilfläche abschneiden. 
Diese sollte am anderen Ende der Fläche wieder ergänzt werden.


\begin{figure}[h!]
  \centering
  \includegraphics[width=0.45\linewidth]{img/burgpark/baustelle.jpg}
  \caption{Baustelle an und auf Blühfläche}
\end{figure}

\end{flushleft}

\newpage

\textbf{Insekten im Burgpark:}

\begin{figure}[h!]
  \centering
  \begin{subfigure}[b]{0.41\linewidth}
    \includegraphics[width=\linewidth]{img/pinselkaefer.jpg}
    \caption{Pinselkäfer}
  \end{subfigure}
  \begin{subfigure}[b]{0.50\linewidth}
    \includegraphics[width=\linewidth]{img/schmalbock.jpg}
    \caption{Rothalsbock}
  \end{subfigure}
  \begin{subfigure}[b]{0.48\linewidth}
    \includegraphics[width=\linewidth]{img/blaeuling.jpg}
    \caption{Hauhechel-Bläuling}
  \end{subfigure}
  \begin{subfigure}[b]{0.43\linewidth}
    \includegraphics[width=\linewidth]{img/weichkaefer.jpg}
    \caption{Ockerbrauner Weichkäfer}
  \end{subfigure}
  \caption{Insekten Burgpark}
\end{figure}

Auf der Blühfläche im Burgpark konnten viele blütenbesuchende Insekten beobachtet werden.

\newpage
\section{Gesamtschule}
\begin{figure}[h!]
  \includegraphics[width=\linewidth]{img/gesamtschule/mai.jpg}
  \caption{Gesamtschule}
\end{figure}

Nach einem eher mageren Start im Jahr 2019 ist die Vegetation auf der Fläche 2020 sehr gut angegangen.

Hier wäre die Überlegung auch auf den angrenzenden Grünflächen des Parkplatzes Blühflächen anzulegen.

Eine Besonderheit war der Nachweis einer auf Grashüpfer lauernde Wespenspinne im Spätsommer im ungemähten Bereich.


\begin{figure}[h!]
  \centering
  \begin{subfigure}[b]{0.34\linewidth}
    \includegraphics[width=\linewidth]{img/gesamtschule/wespenspinne.jpg}
    \caption{Wespenspinne}
  \end{subfigure}
  \begin{subfigure}[b]{0.38\linewidth}
    \includegraphics[width=\linewidth]{img/gesamtschule/grashuepfer.jpg}
    \caption{Feld-Grashüpfer}
  \end{subfigure}
  \caption{Insekten Gesamtschule}
\end{figure}

\clearpage
\section{Sudetenstr.}
\begin{figure}[h!]
  \includegraphics[width=\linewidth]{img/asg/juli.jpg}
  \caption{Sudetenstr. 21. Juli 2020}
\end{figure}
Der Blühstreifen wurde versehentlich im April gemäht, daher gab es hier einen etwas schwierigen Start und eine magere Margeriten-Phase.

Im weiteren Verlauf hat die Vegetation auf der Fläche aber gut aufgeholt, trotz der Trockenheit. 
Es gibt jedoch ein paar sehr trockene Störstellen, insgesamt besteht aber kein Grund zur Sorge. 

Auch die umgebenden Fläche stand teilweise recht hoch. Hier wäre ein Vorschlag des BUNDs zu überlegen, ob statt des schmalen, geschwungenen Blühstreifens die gesamte Fläche in einen Blühstreifen umgewandelt werden sollte und nur an den Rändern zur Verkehrssicherheit gemäht wird.

\clearpage
\section{Frechenerstr.}
\begin{figure}[h!]
  \includegraphics[width=\linewidth]{img/frechenerstr/mai.jpg}
  \caption{Frechenerstr. 29. Mai 2020}
\end{figure}

Der Blühstreifen an sich ist gut entwickelt, geht jedoch etwas unter in der größeren umgebenden hochstehenden Wiese.

Hier wäre es besser, wenn die gesamte Fläche sich zur Blühfläche entwickeln würde.

\clearpage
\section{Sielsdorf}
\begin{figure}[h!]
  \includegraphics[width=\linewidth]{img/sielsdorf/juni.jpg}
  \caption{Sielsdorf 24. Juni 2020}
\end{figure}

Der Blühstreifen zwischen Fahrradweg und Straße ist gut entwickelt und blüht sehr vielfältig.

\clearpage
\section{Heristalstr.}
\begin{figure}[h!]
  \includegraphics[width=\linewidth]{img/heristal/mai.jpg}
  \caption{Heristalstr. 22. Mai 2020}
\end{figure}

Die Fläche ist klein und etwas schattig und wird bedrängt von Brombeeren aus den umgebenden Hecken.
Dennoch ist sie gut entwickelt. Auffällig ist der häufige Rainfarn in der Fläche.

Hier wurde im Juni komplett gemäht, die Fläche war jedoch bald wieder voll in Blüte.

\clearpage
\section{Randkanal}
\begin{figure}[h!]
  \includegraphics[width=\linewidth]{img/randkanal/juni.jpg}
  \caption{Randkanal 24. Juni 2020}
\end{figure}

Die zwei Blühflächen innerhalb der Kompensationsfläche sind gut entwickelt.

Viele Insekten lassen sich dort beobachten, was sicherlich auch an der Größe der Fläche insgesamt liegt und auch an der unversiegelten Umgebung.

\clearpage
\textbf{Insekten am Randkanal:}

\begin{figure}[h!]
  \centering
  \begin{subfigure}[b]{0.44\linewidth}
    \includegraphics[width=\linewidth]{img/randkanal/blauling.jpg}
    \caption{Kurzschwänziger Bläuling}
  \end{subfigure}
  \begin{subfigure}[b]{0.44\linewidth}
    \includegraphics[width=\linewidth]{img/randkanal/streifenwanze.jpg}
    \caption{Larve Streifenwanze}
  \end{subfigure}
  \begin{subfigure}[b]{0.45\linewidth}
    \includegraphics[width=\linewidth]{img/randkanal/distelwickler.jpg}
    \caption{Distelwickler}
  \end{subfigure}
  \begin{subfigure}[b]{0.43\linewidth}
    \includegraphics[width=\linewidth]{img/randkanal/wildbiene.jpg}
    \caption{unbest. Wildbiene}
  \end{subfigure}
  \caption{Insekten Randkanal}
\end{figure}

\clearpage
\section{H.Sürthweg}
\begin{figure}[h!]
  \includegraphics[width=\linewidth]{img/suerthweg/juni.jpg}
  \caption{H.Sürthweg. 24. Juni 2020}
\end{figure}

Auch hier wurde laut Anwohner versehentlich im Frühjahr gemäht.
Die Fläche hat sich später dennoch gut entwickelt. 

\clearpage
\section{Hürther Bogen}
\begin{figure}[h!]
  \includegraphics[width=0.95\linewidth]{img/bogen/april.jpg}
  \caption{Hürther Bogen 28. April 2020}
\end{figure}

Die Fläche leidet insgesamt sehr unter der Trockenheit.
Hier ist es besonders trocken, vermutlich wegen der Hanglage und auch durch die Umgebung aus Straßen, Häusern und Beton. Dies führt zu einer weiteren Aufheizung und Verdunstung.

Evtl. sollte hier eine etwas andere Saatmischung verwendet werden, die mit arideren Bedingungen zurecht kommt.

Trotz allem blühte es auch hier und einige Insekten wurden beobachtet. 

\begin{figure}[h!]
  \centering
  \includegraphics[width=0.45\linewidth]{img/bogen/mai.jpg}
  \caption{Trockener Mai am Hürther Bogen}
\end{figure}


\clearpage
\textbf{Insekten am Hürther Bogen:}

\begin{figure}[h!]
  \centering
  \begin{subfigure}[b]{0.47\linewidth}
    \includegraphics[width=\linewidth]{img/bogen/wildbiene.jpg}
    \caption{Gelbbinden-Furchenbiene}
  \end{subfigure}
  \begin{subfigure}[b]{0.42\linewidth}
    \includegraphics[width=\linewidth]{img/bogen/florfliegenlarve.jpg}
    \caption{Florfliegenlarve}
  \end{subfigure}
  \begin{subfigure}[b]{0.44\linewidth}
    \includegraphics[width=\linewidth]{img/bogen/siebenpunkt.jpg}
    \caption{Siebenpunkt-Marienkäfer}
  \end{subfigure}
  \begin{subfigure}[b]{0.44\linewidth}
    \includegraphics[width=\linewidth]{img/bogen/trapezspinne.jpg}
    \caption{Baldachinspinne}
  \end{subfigure}
  \caption{Insekten Hürther Bogen}
\end{figure}

Die Geldbinden-Furchenbiene ist eine Art warmer trockener Standorte, dies passt zum trockenen Standort der Blühfläche am Hürther Bogen.

\clearpage
\section{Berrenratherstr.}
\begin{figure}[h!]
  \includegraphics[width=\linewidth]{img/berrenrather/august.jpg}
  \caption{Berrenratherstr. 2. August 2020}
\end{figure}

Diese große Fläche bietet einiges an Potenzial, auch im Verbund mit den weiteren Blühflächen in der Nähe.

Durch die späte Aussaat im ersten Jahr war das Resultat noch nicht besonders überzeugend. Üblicherweise wird es im zweiten Jahr besser.

\clearpage
\section{Kendenich Friedhof}
\begin{figure}[h!]
  \includegraphics[width=\linewidth]{img/kendenich/september.jpg}
  \caption{Kendenich Friedhof. 4. September 2020}
\end{figure}

Von den neuen Flächen ist die Aussaat hier am besten angegangen. Aber auch hier ist noch relativ viel Gänsefuß in der Fläche, der hoffentlich im nächsten Jahr im Bestand zurückgehen wird.

\clearpage
\section{Am Lintacker}
\begin{figure}[h!]
  \includegraphics[width=\linewidth]{img/lintacker/april.jpg}
  \caption{Am Lintacker 28. April 2020}
\end{figure}

Hier wurde recht spät eingesät. Daher dominierte hier der Gänsefuß.
Nächstes Jahr sollte sich die Vegetation artenreicher entwickeln.

\clearpage
\section{De Bütt}
\begin{figure}[h!]
  \includegraphics[width=\linewidth]{img/buett/juli.jpg}
  \caption{De Buett}
\end{figure}

Diese neu angelegte Fläche hat wegen einer Baumaßnahme nicht funktioniert.

Im nächsten Jahr sollten wir beobachten, ob dort doch etwas überlebt hat, ggf. müsste hier nochmal eingesät werden.

\clearpage
\section{Aussicht}
Im Jahr 2021 werden wohl erstmal keine neuen Flächen angelegt werden, obwohl es noch Potenzial gäbe.

Zunächst muss noch mal mit allen Beteiligten geprüft werden, wie die Vorgehensweise und Pflege der Flächen funktioniert hat.

Mögliche Erweiterungen an den Flächen sollten jedoch besprochen werden, auch um die Mäharbeiten zu erleichtern.

Der BUND plant für 2021 einige Flächen per Handsense, im Sinne einer schonenden, naturnahen und aktiven Pflege  zu mähen und weitere Kartierungen der Fauna und Flora auf und um die Blühflächen herum durchzuführen.

Die Webseite \href{https://hürth-blüht.de}{https://hürth-blüht.de} soll auch weiterhin gepflegt werden und über Neuigkeiten berichten.

\end{document}
